\documentclass[a4paper]{article}

\title{How to write}
\author{
    Hyosang Kang\\
    \small Division of Mathematics\\ 
    \small School of Interdisciplinary Studies\\ 
    \small DGIST
}
\date{\today}

\begin{document}
\maketitle

% \section{Why do I write?}
% \paragraph{Why Did I Write This Article?}

It was Christmas Eve when I began working on this article. I had been struggling to write another piece on \textit{how to do projects} and \textit{mathematical concepts explained in Go language.} (Admittedly, not easy topics to tackle.)

To clear my mind, I decided to relax in the hot tub. As I sat there, I started reflecting on what was holding me back. The problem was simple: I couldn't write. Or at least, I thought I couldn't. Despite believing I knew how to write, the sentences and paragraphs I produced were a complete mess. My sentence structures were unnecessarily complex, and I even found myself inventing phrases and words that didn't exist in English. (Thanks to ChatGPT, I realized that "a lettered number" isn't a proper phrase for "the word form of a number.")

In the hot tub, I considered how I might start fresh. I returned to the basics of writing--tools I will discuss in the sections that follow. As I visualized myself starting from scratch, things began to click. Then, an idea struck me: I should share this with my students. I decided to document the process of writing this article as I wrote it.

This article is the result of that process.

% \paragraph{What/Why did I write?}
I have written scientific papers, journal articles, books, and short articles like this one. The purpose of writing a scientific paper is straightforward: it communicates the work of my colleagues and me to the public. Similarly, journal articles often serve this purpose. However, journal articles can also target readers seeking educational insights, thus serving the broader audience by providing accessible knowledge.

I primarily write books for educational purposes. 
Most textbooks, especially in scientific fields, focus on presenting well-established facts. 
Writing such books can be relatively straightforward because new editions typically don't differ significantly from older ones--they build on a stable foundation of knowledge.

I don't consider myself a professional writer. 
I write because it is one of the few tangible, productive outcomes of my daily work. 
As a university professor, my primary responsibilities are teaching and research, but much of my time is consumed by activities far removed from writing. 
These activities often don't produce outcomes I can share with others. 
Therefore, dedicating extra effort to write about my teaching and research, though challenging, is essential. 
It transforms my hard work into something tangible and valuable for others.

% \paragraph{Why should you write?}
I assume that most readers of this article are undergraduate students, particularly freshmen. 
Why should students bother to write anything beyond completing class assignments? 
The reasons I write apply to you as well.

Writing has a unique power. 
When you write down your thoughts, they no longer remain confined to your mind--they take on a new, external form. 
Once written, these thoughts begin to communicate with you. 
Your written words become an external perspective that can critique your ideas, suggest improvements, or reshape your arguments. 
The best part? All the progress made in this dialogue remains yours to keep.

Writing helps you think more clearly. 
Until you express your thoughts in words--whether spoken or written--they remain abstract and difficult for others to grasp. 
Often, even your best ideas may seem flawed when scrutinized objectively. 
Think about how dreams can feel entirely logical while you're asleep, yet nonsensical upon waking. 
To make your ideas logical, functional, and effective, you need to articulate them through writing or speech.

% \section{What should I write?}

% \paragraph{What do you care the most?}
The best time to write is now because you're awake. 
When we're awake, we think—and we think within the boundaries of our language. 
This means any thought you have can be written down. Sometimes, thoughts are entangled with feelings, but writing helps separate the two. 
Once they are untangled, you can make choices, decisions, and move forward.

Start by writing about what matters most to you. 
It could be your schedule for the day, money transactions, or a to-do list. 
Maybe it's something you want to share on Instagram, Facebook, or with someone specific. 
Writing clears your mind, helping you relax and let go of lingering thoughts. 
That's why you should always write about what feels most important to you in the moment.

Writing doesn't have to be lengthy—a short essay, a paragraph, or even a single sentence is perfectly fine. 
The key is to make writing enjoyable. 
If it's not interesting to you, it's likely not worth your time.

\vskip 1em
{\centering What do you want to write about? \par}
\vskip 1em

% \section{How can I write?}

Let's get to the point: how can one write? What strategy should one follow? I'll share insights from my own experience writing the article titled ``How to Do Projects.'' Unless stated otherwise, this will be the article I refer to.

When I decided to write it, I had only a vague idea of what I wanted to say. My goal was to help students succeed with their projects and to build their confidence in completing them. I had a few strategies in mind, but they weren't fully developed or organized.

It's perfectly fine to start writing without having every thought or word planned in advance. What matters most is your motivation--\textit{why you want to write}. For me, the motivation was clear: I wanted to support students. From observing their struggles and seeing how they performed in class while presenting their projects, I felt a strong drive to address the gaps and improve their experience.

\vskip 1em
{\centering Why do want to write? \par}
\vskip 1em

The first thing I do when I write is to create a table of contents. 
At this stage, I focus on just three core questions: \textit{What}, \textit{Why}, and \textit{How}. 
Below are the chapter titles I initially drafted (subject to revision as the writing evolves):


\begin{enumerate}
    \item What Are Projects?
    \item Why Do We Do Projects?
    \item How Should We Do Projects?
\end{enumerate}

Once I had the main chapters, I began developing section titles under each chapter. 
This involved answering the primary questions or expanding on them with further inquiries. 
Here's how the structure unfolded:

\begin{enumerate}
    \item What Are Projects?
    \begin{enumerate}
        \item Projects are what identify problems  
        \item Projects are what declare problems  
        \item Projects are what define problems  
        \item Projects are what solve problems  
    \end{enumerate}
    \item Why Do We Do Projects?
    \begin{enumerate}
        \item Projects give meaning to our lives  
        \item Projects create systems to solve problems  
        \item Projects unify individual strengths  
        \item Projects leave traces of our work  
    \end{enumerate}
    \item How Should We Do Projects?
    \begin{enumerate}
        \item How to start a project?  
        \item How to define a project?  
        \item How to manage a project?  
        \item How to finish a project?  
    \end{enumerate}
\end{enumerate}

\end{document}