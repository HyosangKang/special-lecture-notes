\documentclass{beamer}
\usefonttheme{serif}

\usepackage{amsmath,amsthm,amssymb,amsfonts,amscd,mathrsfs,amsxtra,multirow,kotex,mathtools,gensymb,textcomp,lipsum,tikz,verbatim,color,soul,courier,mdframed,xcolor}
\usepackage[normalem]{ulem}
\usetikzlibrary{calc,matrix,arrows,chains,positioning,scopes}
\usepackage{pdfpages}

\theoremstyle{plain}
\newtheorem{thm}{Theorem}[section]
\newtheorem{prop}[thm]{Proposition}

\theoremstyle{definition}
\newtheorem{defn}[thm]{Definition}
\newtheorem{exmp}[thm]{Example}
\newtheorem{excs}[thm]{Exercise}
\newtheorem{rem}[thm]{Remark}
\newtheorem{prob}[thm]{Problem}

\newcommand \tr[1]{\textcolor{red}{#1}}
\newcommand{\tikzmark}[1]{\tikz[overlay,remember picture] \node (#1) {};}
\newcommand{\varep}{\varepsilon}
\newcommand{\DrawBox}[1][]{%
    \tikz[overlay,remember picture]{
    \draw[red,#1]
      ($(left)+(-0.2em,0.9em)$) rectangle
      ($(right)+(0.2em,-0.3em)$);}
}

\newcommand{\tikzmarkk}[2]{
    \tikz[overlay,remember picture,baseline] 
    \node[anchor=base] (#1) {$#2$};
}
\newcommand*\circled[1]{\tikz[baseline=(char.base)]{
            \node[shape=circle,draw,inner sep=2pt] (char) {#1};}}

\tikzset{join/.code=\tikzset{after node path={%
\ifx\tikzchainprevious\pgfutil@empty\else(\tikzchainprevious)%
edge[every join]#1(\tikzchaincurrent)\fi}}}

\tikzset{>=stealth',every on chain/.append style={join},
         every join/.style={->}}
\tikzstyle{labeled}=[execute at begin node=$\scriptstyle,
   execute at end node=$]


\addtobeamertemplate{navigation symbols}{}{%
    \usebeamerfont{footline}%
    \usebeamercolor[fg]{footline}%
    \hspace{1em}%
    \raisebox{2pt}[0pt][0pt]{\insertframenumber/\inserttotalframenumber}
}
\setbeamercolor{footline}{fg=blue}
\setbeamerfont{footline}{series=\bfseries}
\title[]{TP404: Introduction to Quantum Computing}

\author[]{Hyosang Kang\inst{1}}

\institute[]{\inst{1}Division of Mathematics\\ School of Interdisciplinary Studies\\ DGIST}

\date[]{Lecture 01\\
Qubits and Quantum Gates}

\begin{document}

\begin{frame}
\titlepage
\end{frame}

\begin{frame}
\begin{defn}
A \textbf{quantum algorithm} is a computational algorithm 
that runs on a quantum computer using quantum circuits.
\end{defn}
\begin{defn}
A \textbf{quantum error correction} is a quantum codes that detects quantum errors.
\end{defn}
\end{frame}

\begin{frame}
\begin{defn}
A vector space $V$ is a set equipped with a vector addition and scalar multiplication.
\begin{enumerate}
\item (vector addition) $v+w\in V$ for all $v,w\in V$
\item (scalar multiplication) $c\cdot v\in V$ 
for any $v\in V$ and $c\in\mathbb F$ (where $\mathbb F=\mathbb C$ or $\mathbb R$) 
\end{enumerate}
\end{defn}
\begin{defn}
A \textbf{qubit} is a unit of quantum information. Any quantum state of a qubit
is a linear combination of $|0\rangle$ and $|1\rangle$ with \underline{complex} coefficients.
\end{defn}
\begin{rem}
Some uses $\mid\uparrow\rangle$, $\mid\downarrow\rangle$ to represent $|0\rangle$, $|1\rangle$ respectively.
\end{rem}
\end{frame}

\begin{frame}
\begin{defn}
The set of quantum states of the form 
$$|\psi\rangle = \cos\frac{\theta}{2}|0\rangle + e^{i\phi}\sin\frac{\theta}{2}|1\rangle$$
is called the \textbf{Bloch sphere}.
\end{defn}
\begin{rem}
The Bloch sphere presents the state of a qubit in a geometric way.
The $|+\rangle$ and $|-\rangle$ states are defined as follows.
\begin{align*}
|+\rangle &= \frac{1}{\sqrt2}|0\rangle + \frac{1}{\sqrt2}|1\rangle \\
|-\rangle &= \frac{1}{\sqrt2}|0\rangle - \frac{1}{\sqrt2}|1\rangle 
\end{align*}
\end{rem}
\end{frame}

\begin{frame}
\begin{defn}
A map $f:V\to W$ between two vector spaces $V, W$ is called \textbf{linear} if 
	$$f(a_1v_1+a_2v_2) = a_1f(v_1)+a_2f(v_2)$$
for all $a_1,a_2\in \mathbb F$ and $v_1,v_2\in V$.
\end{defn}
\begin{defn}
Given a vector space $V$, the \textbf{dual space} $V^*$ is a vector space 
of all linear maps (called \textbf{functionals}):
	$$V^*=\{v^*\mid v^*:V\to \mathbb R\}.$$
\end{defn}
\end{frame}

\begin{frame}
\begin{defn}
An \textbf{inner product} $\langle,\rangle$ on a vector space $V$ 
is a map $\langle,\rangle:V\times V\to\mathbb R$ satisfying
\begin{enumerate}
	\item $\langle a_1v_1+a_2v_2,v_3\rangle = a_1\langle v_1,v_3\rangle + a_2\langle v_2,v_3\rangle$
	\item $\langle v_1,a_2v_2+a_3v_3\rangle = a_2\langle v_1,v_3\rangle + a_3\langle v_1,v_3\rangle$
\end{enumerate}
for all $a_1,a_2,a_3\in\mathbb F$ and $v_1,v_2,v_3\in V$.
A vector space equipped with an inner product $\langle,\rangle$ 
is called a \textbf{Hilbert space}.
\end{defn}
\end{frame}

\begin{frame}
\begin{defn}
A set of vectors $\{\mathbf e_1,\cdots, \mathbf e_n\}$ in a vector space $V$ 
is called a \textbf{basis} of $V$ if every vector $v\in V$ 
is uniquely written as a linear combination of $\mathbf e_i$'s.
That is, there exists a unique tuple of coefficients $(a_1,\cdots,a_n)\in\mathbb F^n$ such that
	$$v = a_1\mathbf e_1+\cdots+a_n\mathbf e_n.$$
A basis is called \textbf{orthonormal} if 
\begin{enumerate}
	\item the inner product of any pair of 
	two distinct elements is $0$, that is,
		$$\langle \mathbf e_i,\mathbf e_j\rangle,\quad i\neq j,$$
	\item and the inner product of between the same element is $1$, that is,
		$$\langle \mathbf e_i,\mathbf e_i\rangle = 1.$$
\end{enumerate}
\end{defn}
\end{frame}

\begin{frame}
\begin{rem}
Let $\{\mathbf e_1,\cdots,\mathbf e_n\}$ be a orthonormal basis of a Hilbert space $V$.
(In general, a Hilbert space may have infinite basis.)
Define a dual vectors $\mathbf e^j$ that satisfies:
	$$\mathbf e^j(\mathbf e_i) = \begin{dcases}
	0 & i\neq j \\
	1 & i = j \end{dcases}.$$
Given a vector $v = \sum a_i\mathbf e_i\in V$, let $v^*$ be a dual vector defined as
	$$v^* = \sum \overline{a_i}\mathbf e^i.$$
($\overline a$ is the complex conjugate of $a$.)
Then the inner product $\langle v, w\rangle$ between two vectors $v,w\in V$
can be written as follows:
	$$\langle v, w\rangle = v^*(w) = w^*(v).$$
\end{rem}
\end{frame}

\begin{frame}
\begin{defn}
The set of all quantum states of a single qubits is a complex vector space $\mathcal H$
spanned by two vectors, which call \textbf{ket}-vector (simply \textit{ket}):
	$$|0\rangle,\quad |1\rangle.$$
The dual vectors to $|0\rangle$, $|1\rangle$ are called \textbf{bra}-vectors (simply \textit{bra})
and denoted by
	$$\langle 0|, \quad \langle 1|.$$
\end{defn}
\end{frame}

\begin{frame}
\begin{rem}
Traditionally, the dual of a ket $|\psi\rangle$ is denoted by $|\psi\rangle^\dagger$. 
(The symbol $\dagger$ is called \textit{dagger}.) 
The dual of a quantum state $|\psi\rangle = a_0|0\rangle + a_1|1\rangle$ is the following bra:
	$$|\psi\rangle^\dagger = \overline{a_0}\langle 0| + \overline{a_1}\langle 1|.$$
\end{rem}
\begin{defn}
The inner product on $\mathcal H$ is fully determined by the evaluation of \textit{bra}s on \textit{ket}s:
	$$\langle 0|1\rangle = \langle 1|0\rangle = 0$$
	$$\langle 0|0\rangle = \langle 1|1\rangle = 1$$ 
\end{defn}
\end{frame}

\begin{frame}
\begin{defn}
An \textbf{operator} $\Phi$ on a Hilbert space $\mathcal H$ is a linear map to itself: $\Phi:\mathcal H\to\mathcal H$.
\end{defn}
\begin{rem}
Given a orthonormal basis $\{\mathbf e_1,\cdots,\mathbf e_n\}$ on a Hilbert space $\mathcal H$,
an operator $\Phi$ can be written as a formal sum:
	$$\Phi = \sum_{i,j=1}^n a_{ij}\mathbf e_i\otimes\mathbf e^j.$$
The symbol $\otimes$ is called a \textit{tensor product} and here we use 
$\mathbf e_i\otimes\mathbf e^j$ as an operator $\mathcal H\to\mathcal H$ defined as
	$$(\mathbf e_i\otimes\mathbf e^j)(v) = \mathbf e^j(v)\mathbf e_i$$
for all $v\in \mathcal H$.
\end{rem}
\end{frame}

\begin{frame}
\begin{rem}
Any operator on the Hilbert space of a single qubit state can be written as 
a linear combination of \textit{ket}-\textit{bra} products:
	$$\Phi = a_{00}|0\rangle\langle0| + a_{01}|0\rangle\langle1| + a_{10}|1\rangle\langle0| + a_{11}|1\rangle\langle 1|$$
\end{rem}
\begin{defn}
Given an operator $\Phi$ on a Hilbert space $\mathcal H$, 
the dual operator $\Phi^*$ is the operator on $\mathcal H$ satisfying
	$$\langle v,\Phi(w)\rangle = \langle\Phi^*(v),w\rangle$$
for all $v,w\in\mathcal H$.
\end{defn}
\end{frame}

\begin{frame}
\begin{prop}
Given an operator $\Phi = |\phi\rangle\langle \psi|$, the dual operator $\Phi^\dagger$ is written as following:
	$$\Phi^\dagger = (|\phi\rangle\langle \psi|)^* = \langle\psi|^\dagger|\phi\rangle^\dagger.$$ 
\end{prop}
\begin{rem}
For simplicity, we will denote $|\phi\rangle^\dagger = \langle\phi|$ 
and $\langle\psi|^\dagger = |\psi\rangle$ from now on only when there is no ambiguity.
\end{rem}
\end{frame}

\begin{frame}
\begin{defn}
Given two vector spaces $V,W$ over the same field $\mathbb F$,
the \textbf{tensor product} $V\otimes W$ is a vector space consists of all 
vectors of the form $v\otimes w$ for $v\in V$ and $w\in W$ which satisfies the following:
\begin{enumerate}
	\item $c(v\otimes w) = (cv)\otimes w = v\otimes(cw)$;
	\item $(v_1+v_2)\otimes w = v_1\otimes w + v_2\otimes w$;
	\item $v\otimes(w_1+w_2) = v\otimes w_1 + v\otimes w_2$.
\end{enumerate}
\end{defn}
\begin{defn}
Let $\mathcal H$ be the Hilbert space of a single qubit. 
The set of all quantum states of $n$ qubits is represented by $\mathcal H^{\otimes n}$,
	$$\mathcal H^{\otimes n} = \underbrace{\mathcal H\otimes\cdots\otimes\mathcal H}_{n\textrm{ times}}$$
\end{defn}
\end{frame}

\begin{frame}
\begin{rem}
Any quantum state of $n$ qubit is a linear combination of \textit{basis states}:
	$$|e_{n-1}\rangle\otimes\cdots\otimes|e_0\rangle$$
where $e_i = 0,1$.
For simplicity, we write each basis states in binary digits (or natural numbers).
	$$|e_{n-1}\cdots e_0{}_{(2)}\rangle_n.$$
For example, a basis state for 4 qubit system can be written as
	$$|0\rangle\otimes|0\rangle\otimes|1\rangle\otimes|0\rangle = |0010{}_{(2)}\rangle_4 = |2\rangle_4$$
\end{rem}
\end{frame}

\begin{frame}
\begin{defn}
Let $U_i$, $i=0,\cdots n-1$, be an operator acting on the $i$-th qubit.
Then the operator $U_0\otimes\cdots\otimes U_{n-1}$ acts on the $n$ qubit system as follows:
when $|\psi_i\rangle$ is a quantum state for $i$-qubit,
	$$(U_{n-1}\otimes\cdots\otimes U_0)(|\psi_{n-1}\rangle\otimes\cdots\otimes|\psi_0\rangle) = 
	U_{n-1}|\psi_{n-1}\rangle\otimes\cdots U_0|\psi_0\rangle$$
\end{defn}
\begin{exmp}
The operator 
	$$|0\rangle\langle0|\otimes|1\rangle\langle1| + |0\rangle\langle1|\otimes|1\rangle\langle0|$$
acting on the 2 qubit system sends two basis states $|01\rangle$ and $|10\rangle$ to $|01\rangle$
and kills all others.
\end{exmp}
\end{frame}

\begin{frame}
\begin{defn}[Born's rule]
Let $|\psi\rangle = \sum a_i|\phi_i\rangle$ be a quantum state on $n$ qubit system
where each $|\psi_i\rangle = |e_{n-1}^{(i)}\cdots e_0^{(i)}\rangle$ is a basis state.
\begin{enumerate}
	\item If we observe $e_j^{(i)}$ on each $j$-th qubit after the measurement, 
	the state of $n$-qubit is fixed to the basis state $|\psi_i\rangle$.
	\item Suppose $\sum |a_i|^2 = 1$. The probability to observe the state $|\psi_i\rangle$
	after the measurement is $|a_i|^2$.
\end{enumerate}
\end{defn}
\begin{rem}
The value $\sum|a_i|^2$ of a quantum state
$|\psi\rangle = \sum a_i|\psi_i\rangle$ 
is called the \textit{strength} of the state.
From now on, we always assume that the strength
of every quantum state is normalizaed to $1$.
An operator that preserves the strength of any quantum
state is called \textbf{unitary}.
We will always deal with unitary operators.
\end{rem}
\end{frame}

\begin{frame}
\begin{defn}
Given two matrices $A=(A_{ij})$, $B=(B_{ij})$
of sizes $m\times n$ and $p\times q$, respectively,
the \textbf{tensor product} $A\otimes B$ is 
a $mp\times nq$ matrix defined by 
$$A\otimes B = \begin{pmatrix}
	A_{11}B & \cdots & A_{1n}B \\
	\vdots & \ddots & \vdots \\
	A_{m1}B & \cdots &A_{mn}B \end{pmatrix}$$
\end{defn}
\begin{prop}
For $n\times n$ matrices $A, C$ and $m\times m$ matrices $B, D$,
	$$(A\otimes B)(C\otimes D) = AC\otimes BD$$
\end{prop}
\end{frame}

\begin{frame}
\begin{defn}
We can identify the basis states $|0\rangle$ and $|1\rangle$
with $2\times1$-matrix as follows:
	$$|0\rangle\longleftrightarrow\begin{bmatrix}1\\0\end{bmatrix},
	\quad |1\rangle\longleftrightarrow\begin{bmatrix}0\\1\end{bmatrix}.$$
Likewise, the bras $\langle0|$ and $\langle1|$ are identified as follows:
	$$\langle0|\longleftrightarrow\begin{bmatrix}1 & 0\end{bmatrix},
	\quad \langle1|\longleftrightarrow\begin{bmatrix}0 & 1 \end{bmatrix}.$$
The tensor products of quantum states are preserved as matrix tensor product.
\end{defn}
\end{frame}

\begin{frame}
\begin{exmp}
The operator 
	$$|0\rangle\langle1|+|1\rangle\langle0|$$
is identified with $2\times2$ matrix as follows:
	$$\begin{bmatrix}
		1 \\ 0
	\end{bmatrix}\begin{bmatrix}
		0 & 1
	\end{bmatrix} + \begin{bmatrix}
		0 \\ 1
	\end{bmatrix}\begin{bmatrix}
		1 & 0
	\end{bmatrix} = \begin{bmatrix}
		0 & 1 \\ 0 & 0
	\end{bmatrix} + \begin{bmatrix}
		0 & 0 \\ 1 & 0
	\end{bmatrix} = \begin{bmatrix}
		0 & 1 \\ 1 & 0
	\end{bmatrix}$$
\end{exmp}
\begin{rem}
Every computation on quantum states can be translated in terms of matrices
and their tensor product. However, it is often convenient to use 
bra-ket notations in a large system because the size of matrix 
grows \textit{exponentially}.
\end{rem}
\end{frame}

\begin{frame}
\begin{defn}
Let $\mathbf 1$ be the identity operator.
An operator $U$ satisfying $U^\dagger U = \mathbf 1$ is called a \textbf{unitary gate}.
\end{defn}
\begin{defn}
The most important unitary gates acting on a single qubit
are $X$, $Z$, $Y$-gates and the Hadamard gate ($H$) defined as below:
	\begin{align*}
	X &= \begin{bmatrix}
		0 & 1 \\ 1 & 0 
	\end{bmatrix},\quad Z = \begin{bmatrix}
		1 & 0 \\ 0 & -1
	\end{bmatrix},\quad Y = iXZ\\
	H &= \frac{1}{\sqrt2}\begin{bmatrix}
		1 & 1 \\ 1 & -1
	\end{bmatrix}.
	\end{align*}
\end{defn}
\end{frame}

\begin{frame}
\begin{defn}
The controlled-NOT gate uses two qubits,
one for \textit{control} qubit and the other for \textit{target} qubit.
It acts $X$ gate on the target qubit 
when the state of the control qubit is $|1\rangle$.
If the indices of the control and target qubits are $i, j$,
we denote the controlled-NOT gate as $C_iX_j$.
The matrix representation of $C_1X_0$ is
	$$C_1X_0 = \begin{bmatrix}
		1 & 0 & 0 & 0 \\
		0 & 1 & 0 & 0 \\
		0 & 0 & 0 & 1 \\
		0 & 0 & 1 & 0 
	\end{bmatrix}$$
\end{defn}	
\end{frame}

\begin{frame}
\begin{defn}
The SWAP gates interchange the staes of two qubits.
The SWAP gate acting on the $i$-th and $j$-th qubits is 
denoted by $S_{ij}$. The matrix representation of $S_{01}$ is
	$$S_{01} = \begin{bmatrix}
		1 & 0 & 0 & 0 \\
		0 & 0 & 1 & 0 \\
		0 & 1 & 0 & 0 \\
		0 & 0 & 0 & 1 
	\end{bmatrix}$$
\end{defn}
\end{frame}

\begin{frame}
\begin{prop}
The following equalities hold.
\begin{enumerate}
	\item $ZX = -XZ$
	\item $C_iX_j = \displaystyle\frac{1}{2}(\mathbf 1 + Z_i) + \frac{1}{2}X_j(\mathbf 1 - Z_i)$
	\item $S_{ij} = \displaystyle\frac{1}{2}(\mathbf 1 + X_iX_j + Y_iY_j + Z_iZ_j)$
\end{enumerate}
\end{prop}
\end{frame}
\end{document}

